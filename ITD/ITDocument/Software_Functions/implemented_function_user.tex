\subsection{Add height and weight}
In the settings area in the menu it is possible to add height and weight, but not a wearable device. This function has been only half completed because none of the group members have a wearable device. A user must add his/her measures in order to start sending his/her personal health parameters to the server.This function not only follows the working flow of a well known health app called \textit{Health}, but it is also a good base point for the future development of the ASOS service. Given age, sex, weight and height parameters it is possible to implement a good algorithm to evaluate the threshold below which the user can be consider in life danger.

\subsection{Send health parameters to the server}
Although it is not possible to add a wearable device, it has been implemented a thread in which random values are sent to the server as the user's health parameters. This function has been implemented to simulate the DB's filling and in order to to be a good starting point for the future possibility to connect via bluetooth a wearable device to the app.

\subsection{See personal health parameters}
Every user, in his/her personal area, can have access to his/her personal data. This is the users' app main function and, for this reason, it has been implemented in the better possible way by givin the possibility to see not only the daily health parameters but also the weekly, monthly and year ones. \\ The shown health parameters are the minimum, average and maximum heart beat; the minimum and maximum vein pressure and the minimum and maximum temperature detected till that moment.

\subsection{Accept or deny Third Parties data's requests}
This is a collateral function strictly related to one of the most important functions in the other stackholder's app. Because of the third parties' possibility to require single users' data , it has been found necessary to give to the user the possibility to choose to accept or deny the request.

