\def\A {Name}
\def\B {Actors}
\def\C {Entry conditions}
\def\D {Event Flow}
\def\E {Exit conditions}
\def\F {Exceptions}
\def\G {Goals}
\def\H {Requirements}

\renewcommand{\arraystretch}{1.5}

\begin{figure}[h!]
	\includegraphics[width=1.00\textwidth]{./pictures/usecase_diagram.png}\par
	\caption{Figure 1: Use Case Diagram}
\end{figure}

\FloatBarrier

\subsubsection{Sign up}
\begin{center}
	\begin{longtable}{ | p{0.3\textwidth} | p{0.7\textwidth} | }
		\hline 
		 \A &  Sign Up \\ 

		\hline
		 \B &  User - Third party \\ 

		\hline
  		 \C &  The individual or the third party has downloaded the app on his/her smartphone.\\ 

		\hline
		 \D & \begin{enumerate}
			\item The individual opens the app on his/her smartphone;
			\item He/she clicks on the \textit{Register here} link;
			\item Fills all the mandatory fields and provide all his/her personal data;
			\item Clicks on "Register" button;
			\item The system saves all personal data and creates a new personal area.
		\end{enumerate} \\

		\hline
		 \E &  The user or the third party has a personal area and now is able to use the application.\\

		\hline
		 \F & \begin{enumerate}
			\item The user or the third party is already signed up;
			\item The user or the third party doesn't fill all the mandatory fields;
			\item The e-mail has already been registered.
		\end{enumerate} All the exceptions are handled by notifying the user and taking him back to the sign up activity.\\
		
		\hline
		 \G &  G1 - G2\\

		\hline
		 \H &  R1 - R3 - R4 - R6 \\
		\hline

	\end{longtable}
\end{center}

\subsubsection{Login}
\begin{center}
	\begin{longtable}{ | p{0.3\textwidth} | p{0.7\textwidth} | }
		\hline
		 \A &   Login \\ 

		\hline
		 \B &  User - Third party \\ 

		\hline
  		 \C &  The user or the third party has the application installed; has succesfully registered and has a personal area.\\ 

		\hline
		\D & \begin{enumerate}
			\item The user  or the third party opens the app on his/her smartphone;
			\item Writes his/her credentials;
			\item Clicks on "Login" button;
		\end{enumerate} \\

		\hline
		\E & The user  or the third party is succesfully redirected to his/her main page.\\

		\hline
		\F & \begin{enumerate}
			\item The user or the third party doesn't register yet;
			\item The user or the third party enters an invalid Username;
			\item The user or the third party enters an invalid Password;
			\item He/she doesn't fill all fields.
		\end{enumerate} All the exceptions are handled by notifying the user and taking him/her back to the login activity. \\
		
		\hline
		\G & G1 - G2\\

		\hline
		\H & R2 - R3.1- R5 - R6.1 \\
		\hline

	\end{longtable}
\end{center}

\subsubsection{Add settings details}
\begin{center}
	\begin{longtable}{ | p{0.3\textwidth} | p{0.7\textwidth} | }
		\hline
		 \A &   Add settings details\\ 

		\hline
		 \B &  User \\ 

		\hline
  		 \C &  The user has the application installed; has successfully registered; has a personal area and has a device that can be linked to the smartphone.\\ 

		\hline
		\D & \begin{enumerate}
			\item The user opens the app on his/her smartphone;
			\item He/she login;
			\item Opens the Menu by the three points icon in the main scene;
			\item Clicks on the \textit{Settings} link;
			\item Inserts his/her weight and height;
			\item Adds his/her device;
			\item Clicks the \textit{Save} button.
		\end{enumerate} \\

		\hline
		\E & The app saves successfully all personal data and the device starts to send health parameters.\\

		\hline
		\F & \begin{enumerate}
			\item The user exits from the Settings area without saving, this exception is handled by notifying it to the user;
			\item The user doesn't link any personal device, this exception can't be handled but the user notices it because 					without a linked device the app can't run correctely.
		\end{enumerate} \\
		
		\hline
		\G & G1\\

		\hline
		\H & R3.1 - R3.3 \\
		\hline

	\end{longtable}
\end{center}

\subsubsection{Monitor own parameters}
\begin{center}
	\begin{longtable}{ | p{0.3\textwidth} | p{0.7\textwidth} | }
		\hline
		 \A &  Monitor his/her own parameters\\ 

		\hline
		 \B &  User \\ 

		\hline
  		 \C &    The user has the application installed; has succesfully registered; has a personal area, has added settings details and has a device linked to his/her smartphone.\\

		\hline
		\D & \begin{enumerate}
			\item The user login;
		\end{enumerate} \\

		\hline
		\E & The app opens the main scene in which the user can see his/her health paramters by graphs and numbers.\\

		\hline
		\F & - \\
		
		\hline
		\G & G7\\

		\hline
		\H & R20\\
		\hline

	\end{longtable}
\end{center}

\subsubsection{Receive a third party's data request}
In the figure 3.21 there are three use cases which derives from this abstract one; their difference is given by the last step of the event flow and the exit condition. Because of this reason we have chosen to compress them in just one table.
\begin{center}
	\begin{longtable}{ | p{0.3\textwidth} | p{0.7\textwidth} | }
		\hline
		 \A &   Receive a third party's data request\\ 

		\hline
		 \B &  User \\ 

		\hline
  		 \C &  The user has the application installed; has succesfully registered; has a personal area and has a device linked to his/her smartphone.\\ 

		\hline
		\D & \begin{enumerate}
			\item The user receives a notification on his/her smartphone;
			\item He/she login and finds an alert;
			\item Reads the alert;
			\item Click on the \textit{Accept} button or on the \textit{Refuse} button or on the \textit{Exit} icon.
		\end{enumerate} \\

		\hline
		\E & The allert is closed and the choise is saved: \begin{enumerate}
			\item if he/she has clicked on the \textit{Exit} icon the notification is added in the \textit{Data Request Notification} 				area;
			\item if he/she has clicked on the \textit{Accept} button the third party is added in the \textit{Third Party} area;					\item else it is delated.
		\end{enumerate}\\

		\hline
		\F & \begin{enumerate}
			\item The user doesn't accept neither refuse before the time bound.
		\end{enumerate} This exception is resolved by considering the non answer as a negative one. \\
		
		\hline
		\G & G4\\

		\hline
		\H & R3.1 - R9 - R10 \\
		\hline

	\end{longtable}
\end{center}

\subsubsection{Receive a notification (third party)}
In the figure 3.21 there are three use cases which derives from this abstract one; their difference is given by the last step of the event flow and the exit condition. Because of this reason we have chosen to compress them in just one table.
\begin{center}
	\begin{longtable}{ | p{0.3\textwidth} | p{0.7\textwidth} | }
		\hline
		 \A &   Receive a notification\\ 

		\hline
		 \B &  Third Party \\ 

		\hline
  		 \C &  The third party has the application installed; has succesfully registered and has a personal area.\\ 

		\hline
		\D & \begin{enumerate}
			\item The third party receives a notification;
			\item The third party login and finds an alert;
			\item Reads the alert;
			\item Clicks on the \textit{Accept} button or on the \textit{Refuse} button or on the \textit{Exit} icon.
		\end{enumerate} \\

		\hline
		\E & The allert is closed and the choise is saved: \begin{enumerate}
			\item if the third party  has clicked on the \textit{Accept} button the single user data or the group of users data are 				added in the main scene and the third party will always receive new data ;					
			\item  if the third party  has clicked on the \textit{Refuse} button the single user data or the group of users data are 				added in the main scene and the third party won't receive new data ;	
			\item nothing happens.
		\end{enumerate}\\

		\hline
		\F & -\\
		\hline
		\G & G5 - G6\\

		\hline
		\H & R6.1 - R11 - R12 - R13 - R14 - R15 - R16\\
		\hline

	\end{longtable}
\end{center}

\subsubsection{Subscribe to \textit{ASOS}}
\begin{center}
	\begin{longtable}{ | p{0.3\textwidth} | p{0.7\textwidth} | }
		\hline
		 \A &   Subscribe to \textit{ASOS}\\ 

		\hline
		 \B &  User \\ 

		\hline
  		 \C &  The user has the application installed; has succesfully registered; has a personal area and has a device linked to the smartphone.\\ 

		\hline
		\D & \begin{enumerate}
			\item He/she login;
			\item Opens the menu and clicks on the \textit{ASOS} link;
			\item Checks the check boxes;
			\item Clicks on the \textit{Save} button.
		\end{enumerate} \\

		\hline
		\E & The app saves the choise and activate the \textit{ASOS} service.\\

		\hline
		\F & - \\
		
		\hline
		\G & G8 - G9\\

		\hline
		\H & R17 - R18 - R19 \\
		\hline

	\end{longtable}
\end{center}

\subsubsection{Data request}
\begin{center}
	\begin{longtable}{ | p{0.3\textwidth} | p{0.7\textwidth} | }
		\hline
		 \A &  Request data\\ 

		\hline
		 \B &  Third Party \\ 

		\hline
  		 \C &  The third party has the application installed; has succesfully registered and has a personal area.\\

		\hline
		\D & \begin{enumerate}
			\item The third party login;
			\item Clicks on the plus icon in the main scene: if the request is for single user data the first one else the second;
			\item Fills all the mandatory fields with the required data  ;
			\item Clicks on the \textit{Save and Send request} button.
		\end{enumerate} \\

		\hline
		\E & The app saves the request and starts to elaborate it.\\

		\hline
		\F & \begin{enumerate}
			\item The third party doesn't fill any field.
		\end{enumerate} The exception is handled by notifying the third party and taking him back to the Request Data activity. \\
		
		\hline
		\G & G3 - G3.1 - G3.2\\

		\hline
		\H & R6.2 - R7 - R8 - R8.1 \\
		\hline

	\end{longtable}
\end{center}

\subsubsection{Monitor users' parameters}
\begin{center}
	\begin{longtable}{ | p{0.3\textwidth} | p{0.7\textwidth} | }
		\hline
		 \A &  Monitor users' parameters\\ 

		\hline
		 \B &  Third Party \\ 

		\hline
  		 \C &  The third party has the application installed; has succesfully registered; has a personal area and has received data of some single users and some group of users.\\

		\hline
		\D & \begin{enumerate}
			\item The third party login;
			\item Clicks on a \textit{Fiscal code X} button or on a \textit{Group X} button;
		\end{enumerate} \\

		\hline
		\E & The app opens the health parameter of the user associated to the clicked fiscal code or to the group of users associated to the considered group.\\

		\hline
		\F & - \\
		
		\hline
		\G & G6\\

		\hline
		\H & R6.1 - R15 \\
		\hline

	\end{longtable}
\end{center}